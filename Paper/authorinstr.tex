\documentclass{article}
\usepackage{bnaic}




\title{\textbf{\huge Traffic sucks\\ Does memory help to make it run smoother?}}


\author{Wouter Reckman \affila \and
    Matthia Sabatelli \affilb \and
    Gereon Vienken \affila}


\pagestyle{empty}

\begin{document}
\ttl
\thispagestyle{empty}


\begin{abstract}

\end{abstract}


\section{Introduction}

\subsection{Problem}
The Problem we try to address with our project is the inefficiencies in traffic. Even though many people don't think about it, traffic is an important issue in our society. Many studies have modelt the flow of traffic under various conditions and measured ist relative impact. This paper tries to add to this reservour by trying to understand the impact of driver memory on repeating traffic situations. We all know the situation of drivers that face traffic jams on thier daily commute. But how come these drivers face jams at the same junction every day and still decide to drive the same route? And would it help if they decided to try their luck of the beaten path next time? This article tries to get closer to an answer for these questions. 

\subsection{State of the art}
Our apporach was inspired by an article of Gehrke et al.\cite{gehrke2008traffic}. In this article the authors compared two microscopic traffic simulations. One with a selfish and one with a selfless driver model. The selfish driver simply tried to maxemise his own speed, while the selfless driver used a Markov decition model to maximize the speed over all cars. In the same way as in this article our approach tries to verify the impact of another human trade in traffic simulations, namely the memory aspect.  

\subsection{New Idea}
The new idea our project proposal is to see if memory can actually have an impact on the flow of traffic over longer periods of time. This may also deal with theory of mind but we are not sure at the moment if we will have enough time to also implement a working version that takes this into account. We hope that our implementation can show that with the se of memory to avoid previously encountered situations, the traffic flow over the map is going to go smoother and that the average traveling speed per car increases. In contrast to Gehrke we only use selfish agents, but by making them more intelligent to persue their own goals we hope that they will indirectly help other agents in the process as well.


\section{Method}

  \subsection{Simulation Method}
To make our trafic simulation we have programmed a grid of nodes. Each node represents a junction of streets, the streets are represented by simple lists. Each junction and street are allowed to hold a specific number of cars passing through them with a first-in-first-out schema. A traffic jam in this situation would occur if cars would first try to enter a junction from more directions and then will try to leave it, turning that junction into a kind of bottleneck. This aspect of the traffici is on what our whole simulation goes about. The cars on the road a re-controlled by seperate agents, which give them a spawn location and a destination, these agents can then determin the fastest route with an weighted a* implementation algorithm. The weights of a* can be influenced by the memory of previous traffic jams the agents have encountered. This memory consists of a global array that holds the roads where traffic jams have occured in previous situations. For now we have decided to go with a global memory model since on side it's easier to implement and on the other side it reflects the fact that these information are available for evey driver that deals with a navigation system. 

  \subsection{Experiment Design}
For

\section{Results}

	\subsubsection{Experimental Results}
	
	\subsubsection{Interpretation}
  

\section{Conclusion}

	\subsubsection{Discussion}

	\subsubsection{Relevance}

\bibliographystyle{plain}
\bibliography{mybibfile}



\end{document}








