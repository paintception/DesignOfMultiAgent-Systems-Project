\documentclass[a4paper,hidelinks]{article}
\usepackage{bnaic}
\usepackage[T1]{fontenc}
\usepackage[utf8]{inputenc}

\usepackage{filecontents}
\usepackage{hyperref}
\usepackage{graphicx}


\title{\textbf{\huge DMAS Project report\\ <Project Title>}%
}
\author{Gereon Vienken (s2738805) \and
    Matthia Sabatelli (s2847485) \and
    Wouter Reckman (s2231166)
}
\date{}

\pagestyle{empty}

\begin{document}
\ttl
\thispagestyle{empty}

%\bibliography{dmas-gereon_matthia_wouter-traffic_memory-report}


\begin{abstract}
\noindent
This is the abstract of my paper. Please start the first paragraph of your abstract with a \verb+\noindent+ command.
...
\end{abstract}


\section{Introduction}
Citation \cite{gabel2012cooperative}.

\subsection{Problem}
\subsection{State of the art}
\subsection{New idea}
Add memory. Why would memory help? Unlike in the paper we assume that in reality people will not behave cooperative. They will however, remember when and where the roads are busy and try to avoid those locations.


\section{Method}
\subsection{Simulation model}
\subsection{Experiment design}


\section{Results}
\subsection{Experiment findings}
TODO: how do we want to present results? (graphs, avg. flow/speed/fuel/jams)

\subsection{Interpretation of findings}


\section{Conclusion}
\subsection{Discussion}
\subsection{Relevance}


\bibliographystyle{plain}
\bibliography{dmas-gereon_matthia_wouter-traffic_memory-report}

\begin{filecontents}{dmas-gereon_matthia_wouter-traffic_memory-report.bib}
@article{gabel2012cooperative,
  title={The cooperative driver: Multi-agent learning for preventing traffic jams},
  author={Gabel, Thomas and Riedmiller, Martin},
  journal={International journal of traffic and transportation engineering},
  volume={1},
  number={4},
  pages={67--76},
  year={2012},
  publisher={Scientific \& Academic Publishing}
}
\end{filecontents}

\end{document}
